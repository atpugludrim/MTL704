% Created 2022-10-29 Sat 09:55
% Intended LaTeX compiler: pdflatex
\documentclass[11pt]{article}
\usepackage[utf8]{inputenc}
\usepackage[T1]{fontenc}
\usepackage{graphicx}
\usepackage{grffile}
\usepackage{longtable}
\usepackage{wrapfig}
\usepackage{rotating}
\usepackage[normalem]{ulem}
\usepackage{amsmath}
\usepackage{textcomp}
\usepackage{amssymb}
\usepackage{capt-of}
\usepackage{hyperref}
\title{}
\hypersetup{
 pdfauthor={Mridul Gupta},
 pdftitle={},
 pdfkeywords={},
 pdfsubject={},
 pdfcreator={Emacs 27.1 (Org mode 9.3)}, 
 pdflang={English}}
\begin{document}

\noindent \textbf{Definition - Conjugate Directions}. Let $Q$ be an $n\times n$
matrix and $Q\succ 0$. Any two non-zero vectors (directions) $d^{(1)}$, $d^{(2)}$
$\in\mathbb{R}^n$ are said to be \textit{conjugate vectors} or \textit{conjugate
directions} with respect to $Q$, if $(d^{(1)})^TQd^{(2)}=0$.
\vskip0.5cm\noindent If $Q=I$, conjugacy reduces to orthogonality. More than two vectors
are $Q$-conjugate if they are all mutually $Q$-conjugate.
\vskip0.5cm\noindent \textbf{Result}. Let $\{d^{(0)},d^{(1)},\dotsc,d^{(k)}\}$ be a set
of $k+1$ non-zero vectors which are conjugate with respect to a given positive
definite matrix $Q$. Then the vectors $d^{(0)},d^{(1)},\dotsc,d^{(k)}$ are linearly
independent.
\vskip0.5cm\noindent \textbf{Proof}. To prove, we have to show that
$\alpha_0d^{(0)}+\alpha_1d^{(1)}+\cdots+\alpha_kd^{(k)}=0$ implies that each
$\alpha_j=0$.
Consider:
\[\alpha_0d^{(0)}+\alpha_1d^{(1)}+\cdots+\alpha_kd^{(k)}=0\]
and pre-multiply both sides by $(d^{(j)})^TQ$. Then by conjugacy, it reduces to
\[\alpha_j((d^{(j)})^TQd^{(j)})=0\]
But, since $Q\succ 0$, $(d^{(j)})^TQd^{(j)}>0$. Thus, $\alpha_j=0$. We can do this
for any $j$, and thus prove all $\alpha_j$'s are 0.
\par\hfill$\Box$
\vskip0.5cm For the problem $\min\frac{1}{2}x^TQx-b^Tx$ where $Q\succ 0$ and
$Q^T=Q$, note that at $x^*$, $Qx^*=b$ by setting gradient to 0. And $x^*=Q^{-1}b$.
We can solve this system of linear equations to find $x^*$, but we wanna do it without
explicitly finding $Q^{-1}$ so that it can be applied to general function where
a $Q$ might not exist. We gone prove that given $n$ non-zero $Q$-conjugate vectors
we can find $x^*$ without having to compute $Q^{-1}$.
\vskip0.5cm \noindent \textbf{Result}. Let $d^{(0)},d^{(1)},\dotsc,d^{(n-1)}$ be
a set of $n$ non-zero vectors in $\mathbb{R}^n$ which are conjugate with respect
to $Q$. Then $x^*$, which is the unique solution to the system $Qx=b$ is given by
\[x^*=\sum_{k=0}^{n-1}\left(\dfrac{(d^{(k)})^Tb} {(d^{(k)})Qd^{(k)}} \right)d^{(k)}\]
%
\vskip0.5cm\noindent\textbf{Proof}. $d^{(0)},d^{(1)},\dotsc,d^{(n-1)}$ are
$n$ linearly independent vectors in $\mathbb{R}^n$. \break Or, equivalently, they form
a basis for $\mathbb{R}^n$. Therefore, there exist scalars $\alpha_0,\alpha_1,
\dotsc\alpha_{n-1}$ such that:
\[x^*=\alpha_0d^{(0)}+\alpha_1d^{(1)}+\cdots+\alpha_{n-1}d^{(n-1)}\]
Now, pre-multiply with $(d^{(k)})^TQ$ and use conjugacy to determine $\alpha_k$:
\[\alpha_k=\dfrac{(d^{(k)})^TQx^*}{(d^{(k)})^TQd^{(k)}}\]
but since $x^*$ is solution of equation $Qx=b$, $Qx^*=b$.
\par\hfill$\Box$
%
\vskip0.5cm \noindent We will now rethink this as an iterative process:
\\\textbf{Conjugate Direction Theorem}. Let $\{d^{(0)},d^{(1)},\dotsc,d^{(n-1)}\}$
be a set of $n$ non-zero vectors in $\mathbb{R}^n$ which are conjugate with respect
to $Q$. For any $x^0\in\mathbb{R}^n$, the sequence $\{x^k\}$ generated according to
\begin{align*}
    x^{k+1}&=x^k+\alpha_kd^{(k)},\\
    \alpha_k&=-\dfrac{(g^k)^Td^{(k)}}{(d^{(k)})^TQd^{(k)}},\\
    g^k&=Qx^k-b,
\end{align*}
converges to the unique solution $x^*$ of the system $Qx=b$ exactly after $n$ steps,
i.e. $x^n=x^*$.
%
\vskip0.5cm\noindent\textbf{Proof}. Since $d^{(0)},d^{(1)},\dotsc,d^{(n-1)}$ form
a basis of $\mathbb{R}^n$, there exist scalars \(\lambda_0,\lambda_1,\dotsc, \lambda_{n-1}\)
such that:
\[x^*-x^0=\lambda_0d^{(0)}+\lambda_1d^{(1)}+\dotsc+\lambda_{n-1}d^{(n-1)}\]
Again, pre-multiplying with $(d^{(k)})^TQ$, we get
\[\lambda_k=\dfrac{(d^{(k)})^TQ(x^*-x^0)}{(d^{(k)})^TQd^{(k)}}\]
And, by following the iterative scheme:
\begin{align*}
    x^1-x^0&=\alpha_0d^{(0)}\\
    x^2-x^0&=x^1+\alpha_1d^{(1)}-x^0=\alpha_0d^{(0)}+\alpha_1d^{(1)}\\
    \vdots\quad&\\
    x^k-x^0&=\alpha_0d^{(0)}+\alpha_1d^{(1)}+\dotsc+\alpha_{k-1}d^{(k-1)}\\
\end{align*}
Pre-multiplying by $(d^{(k)})^TQ$:
\begin{equation}(d^{(k)})^TQ(x^k-x^0)=0\label{eq:1}\end{equation}
Now,
\[\lambda_k=\dfrac{(d^{(k)})^TQ(x^*-x^k+x^k-x^0)}{(d^{(k)})^TQd^{(k)}}\]
but by~\ref{eq:1}:
\begin{align*}
    \lambda_k&=\dfrac{(d^{(k)})^TQ(x^*-x^k)}{(d^{(k)})^TQd^{(k)}}\\
    &=\dfrac{(d^{(k)})^T(Qx^*-Qx^k)}{(d^{(k)})^TQd^{(k)}}\\
    &=\dfrac{(d^{(k)})^T(b-Qx^k)}{(d^{(k)})^TQd^{(k)}}\\
    &=\dfrac{(d^{(k)})^T(-g^k)}{(d^{(k)})^TQd^{(k)}}\\
    &=\alpha_k\\
\end{align*}
Also, $x^n-x^0=\alpha_0d^{(0)}+\alpha_1d^{(1)}+\dotsc+\alpha_{n-1}d^{(n-1)}$. But
$\alpha_k=\lambda_k\Rightarrow x^n-x^0=\lambda_0d^{(0)}+\lambda_1d^{(1)}+\dotsc+\lambda_{n-1}d^{(n-1)}
= x^*-x^0$. Thus, $x^n=x^*$.
\par\hfill$\Box$

\section*{Conjugate gradient method for Quadratic case}
\begin{enumerate}
    \item[Step 1] Choose $x^0\in\mathbb{R}^n$, $d^{(0)}=-g^0=b-Qx^0$. Set $k=0$.
    \item[Step 2] Use the scheme:
        \begin{align*}
            x^{k+1}&=x^k+\alpha_kd^{(k)}\\
            \alpha_k&=\dfrac{-(g^k)^Td^{(k)}}{(d^{(k)})^TQd^{(k)}}\\
            d^{(k+1)}&=-g^{k+1}+\beta_kd^{(k)}\\
            \beta_k&=\dfrac{(g^{k+1})^TQd^{(k)}}{(d^{(k)})^TQd^{(k)}}\\
            g^k&=Qx^k-b
        \end{align*}
    \item[Step 3] Continue till we get $x^n=x^*$.
\end{enumerate}
Note that we do not need all the conjugate directions from the start, we are generating
them using $\beta_k$. $\beta_k$ is such that new direction is conjugate to all 
previous directions: $(d^{(k)})^TQd^{(i)}=0,\;\forall\;i=0,1,\dotsc,k$. This can
be seen simply by pre-multiplying $d^{(k+1)}$ with $(d^{(k)})^TQ$ for $i=k$ and
by induction for the rest of $i$.

\vskip0.5cm\noindent\textbf{Result}. For conjugate gradient method, following hold:
\begin{itemize}
    \item $(g^{k+1})^Td^{(k)}=0$\\
    \item $\alpha_k=\dfrac{(g^k)^Tg^k}{(d^{(k)})^TQd^{(k)}}$\\
    \item $\beta_k=\dfrac{(g^{k+1})^Tg^{k+1}}{(g^k)^Tg^k}$\\
    \item $(d^{(k)})^TQd^{(i)}=0, (i=0,1,\dotsc,k-1)$\\
\end{itemize}
%
%
\section*{General unconstrained problem}
$\alpha_k$ is obtained from minimization of $\phi(\alpha)=f(x^k+\alpha d^{(k)})$.
And after every $n-1$ iterations, the process is restarted (Powell's correction).
The direction is reset to the steepest descent direction.\par
%
How is $\alpha_k$ obtained? Here's the sketch:
\begin{itemize}
    \item We know that $\alpha=\dfrac{(d^{(k)})^Td^{(k)}}{(d^{(k)})^TQd^{(k)}}$.
    \item Begin with an estimate $\hat{\alpha}_k$.
    \item Determine $\hat{f}\equiv f(x^k+\hat{\alpha}_kd^{(k)})$.
    \item Now, approximate $f(x)$ around $x=x^k$ with Taylor's series and
        evaluate at $x^k+\hat{\alpha}_kd^{(k)}$:
        \begin{align*}
            f(x)&=f(x^k)+(\nabla f(x^k))^T(x-x^k)+\frac{1}{2}(x-x^k)^T%
            \nabla^2f(x^k)(x-x^k)\\
            f(x^k+\hat{\alpha}_k)&=f(x^k)+(\nabla f(x^k))^T(\hat{\alpha}_kd^{(k)})%
            +\frac{1}{2}(\hat{\alpha}_kd^{(k)})^T\nabla^2f(x^k)(\hat{\alpha}_kd^{(k)})\\
            %
            \hat{f}&\approx f(x^k)+\hat{\alpha}_k(d^{(k)})^T\nabla f(x^k)+%
            \frac{1}{2}\hat{\alpha}_k(d^{(k)})^T\nabla^2f(x^k)d^{(k)}\\
        \end{align*}
    \item From the actual value of $\hat{f}$ and approximated value of $\hat{f}$
        we can produce an estimate for the denominator in calculation of $\alpha$:
        \[(d^{(k)})^T\nabla^2f(x^k)d^{(k)} = 2\dfrac{\hat{f}-f(x^k) -\hat{\alpha}_k(d^{(k)})^T\nabla f(x^k)}%
        {\hat{\alpha}_k^2}\]
    \item Use this to find $\alpha$.
\end{itemize}
\end{document}
