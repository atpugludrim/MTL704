% Created 2022-09-19 Mon 17:05
% Intended LaTeX compiler: pdflatex
\documentclass[11pt]{article}
\usepackage[utf8]{inputenc}
\usepackage[T1]{fontenc}
\usepackage{graphicx}
\usepackage{grffile}
\usepackage{longtable}
\usepackage{wrapfig}
\usepackage{rotating}
\usepackage[normalem]{ulem}
\usepackage{amsmath}
\usepackage{textcomp}
\usepackage{amssymb}
\usepackage{capt-of}
\usepackage{hyperref}
\author{Mridul Gupta}
\date{\today}
\title{}
\hypersetup{
 pdfauthor={Mridul Gupta},
 pdftitle={},
 pdfkeywords={},
 pdfsubject={},
 pdfcreator={Emacs 27.1 (Org mode 9.3)}, 
 pdflang={English}}
\begin{document}
\pagenumbering{gobble}
\section{First pass}

At iteration \(k\) (crosscheck this),
\begin{align*}
    &P:\min_{\alpha\in\mathbb{R}}\;\phi(\alpha)\\
    &\mbox{where }\phi(\alpha)=x_k-\alpha\nabla f(x)\lvert_{x=x_k}\\
    &\\
    &\mbox{and then}\\
    &\\
    &\hat{\alpha_k}=\underset{\alpha\in\mathbb{R}}{\operatorname{argmin}}\;\phi(\alpha)\\
    &x_{k+1}=x_k-\hat{\alpha_k}\nabla f(x)\lvert_{x=x_k}
\end{align*}
\vspace{1cm}

How to do line search?

\begin{enumerate}
    \item Start with a bracket.
    \item How? Go forward and backward.
    \item Once we have \([a,b]\), we can do golden section or fibonacci section method.
    \item We have \(I_1\) from \([a,b]\). We have to pick an \(\varepsilon\) and then calculate \(n\) from it.
    \item From \(n\), calculate \(F_n\). Write function to calculate \(p_j\) and \(q_j\). Write function to select left interval or right interval.
    \item Details in notes.
\end{enumerate}

\section{Second pass}

Exact steps of forward and backward:
\begin{enumerate}
    \item Start with \(\alpha_0\) and an \(h\) (baby steps, so \(h\) should be
        small value).
    \item Go to \(\alpha_0+h\), see if \(\phi(\alpha_0)>\phi(\alpha_0+h)\).
    \item If yes, go to \(2h\), \(4h\), \(8h\), and keep checking same condition.
    \item If no, then revert backwards using a different GP (or for simplicity
        use the same GP.)
    \item As long as the function is decreasing you keep going forward.
    \item Then as long as the function is decreasing you keep going forward.
    \item You end up with a small bracket where there should be a minima.
\end{enumerate}

Exact steps of fibonacci method:
\begin{enumerate}
    \item \(I_n=\dfrac{I_1}{F_n}\)
    \item \(I_n < \varepsilon\)
    \item \(I_k = I_{k+1}+I_{k+2} = (F_{n-k}+F_{n-k-1})I_n = F_{n-k+1}I_n\)
    \item \(I_{k+2} = I_k - I_{k+1}\)
    \item Either \(x_p^k=x_u^k-I_{k+1}\) or \(x_q^k=x_l^k+I_{k+1}\)
    \item Last mei \(x_p^k = x_q^k\), then use a \(\delta\)-disturbance.
    \item For numerical reasons this can happen before, to \(\delta\) wala ek
        iteration chalaya jayega.
    \item Choose \(\dfrac{\delta}{2}<\dfrac{I_1}{2F_n}\)
    \item See \href{https://en.wikipedia.org/wiki/File:GoldenSectionSearch.png}{image}
        for when to choose which interval.
    \item Due to numerical issues, at some point \(x_p^k\) might be \(>x_q^k\).
        In such case, choose \(x^*\) to be the mid point of \(x_l^k\) and \(x^k_u\).
\end{enumerate}

\section{The third idea}
Since we are doing quadratic optimization, we can find a closed form solution for
\(\alpha\):
\begin{align*}
    &\phi(\alpha)= x^k-\alpha\nabla f(x^k)\\
    &\mbox{and }f(x)=\dfrac{1}{2}x^TQx-b^Tx\\
    &\hat{\alpha_k}\mbox{ is the minimizer of }\phi(\alpha)\\
    &\mbox{setting }\phi'(\alpha)=0\\
    &\nabla f(x)=Qx-b\\
    &\\
    &\phi'(\alpha)=\nabla f\left(x^k-\alpha\nabla f(x^k)\right)^T\nabla f(x^k)\\
    &\\
    &\mbox{let $g=\nabla f(x^k)$ and using $x$ instead of $x^k$ in the following for
    simpler notation}\\
    &\Rightarrow \left(Qx-\alpha Qg-b\right)^T(Qx-b)=0\\
    &\Rightarrow \left(x^TQ^T-\alpha g^TQ^T -b^T\right)(Qx-b)=0\\
    &\Rightarrow x^TQ^TQx-x^TQ^Tb-\alpha g^TQ^TQx + \alpha g^TQ^Tb - b^TQx +\lVert b\rVert^2=0\\
    &\\
    &\mbox{Note: $Q$ is symmetric pd and $x^TQ^Tb=b^TQx$ (transpose of as scalar)}\\
    &\Rightarrow x^TQ^2x-2x^TQb-\alpha g^TQ^2x+\alpha g^TQb+\lVert b\rVert^2=0\\
    &\Rightarrow \alpha=\dfrac{x^TQ^2x-2x^TQb+\lVert b\rVert^2}{g^TQ^2x-g^TQb}\\
    &\\
    &\mbox{given that the denominator is not zero (it is a scalar)}\\
    &\mbox{The denominator is zero only when either the gradient is zero or $Qx=b$}\\
    &\mbox{both of which only happen at the optimum point}\\
    &\mbox{(because }g^TQ(Qx-b)=0\mbox{ only when either $g=0$ or $Qx-b=0$)}\\
    &\\
    &\mbox{Note: $x$ here is not a variable, but actually $x^k$ (a fixed value)}
\end{align*}

\section{Steepest Descent}

\end{document}
